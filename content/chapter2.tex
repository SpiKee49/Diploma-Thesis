\chapter{Analýza JavaScriptových frameworkov z hľadiska ochrany proti XSS}

\section{Úvod do problematiky}
Moderné JavaScriptové frameworky vyvinuli rôzne stratégie na ochranu pred Cross-Site Scriptingom (XSS), čo predstavuje jednu z najrozšírenejších bezpečnostných hrozieb vo webovom vývoji. Tieto frameworky sa snažia nájsť rovnováhu medzi bezpečnosťou a flexibilitou, pričom každý pristupuje k problému s odlišnou filozofiou. V tejto kapitole podrobne analyzujeme, ako hlavné front-endové frameworky riešia túto problematiku, s dôrazom na ich architektonické rozhodnutia a efektivitu pri prevencii XSS útokov.

\section{React: Deklaratívna ochrana s únikovými ventilmi}
React, ako jeden z najpopulárnejších front-endových frameworkov, implementuje bezpečnostný model založený na automatickom escapovaní všetkého obsahu vloženého do JSX šablón. Keď vývojár vloží dynamický obsah pomocou syntaxe \texttt{\{userInput\}}, React automaticky escapuje všetky potenciálne nebezpečné znaky, čím zabraňuje ich interpretácii ako HTML alebo JavaScript. Tento mechanizmus je implementovaný priamo v React DOM knižnici a funguje pre všetky textové obsahy a atribúty.

Napríklad, ak by používateľský vstup obsahoval skript \texttt{<script>alert('XSS')</script>}, React by tento kód zobrazil ako čistý text, nie ako vykonateľný JavaScript:

\begin{lstlisting}[language=JavaScript]
const userInput = "<script>alert('XSS')</script>";
return <div>{userInput}</div>; // Bezpecne, obsah je escapovany
\end{lstlisting}

Avšak React poskytuje aj explicitný spôsob, ako túto ochranu obísť pomocou špeciálneho atribútu \texttt{dangerouslySetInnerHTML}. Tento názov slúži ako varovanie pre vývojárov, že používajú potenciálne nebezpečnú funkcionalitu:

\begin{lstlisting}[language=JavaScript]
const userInput = "<b>tucny text</b>";
return <div dangerouslySetInnerHTML={{__html: userInput}} />; // Potencialne nebezpecne
\end{lstlisting}

Ďalšou oblasťou potenciálnych problémov sú JavaScriptové URL v atribútoch ako \texttt{href}, kde React neposkytuje úplnú ochranu:

\begin{lstlisting}[language=JavaScript]
const userUrl = "javascript:alert('XSS')";
return <a href={userUrl}>Klikni sem</a>; // Potencialne nebezpecne
\end{lstlisting}

React tiež neposkytuje komplexnú ochranu proti DOM-based XSS útokom, keďže mnohé takéto útoky sa môžu vyskytnúť mimo Reactového ekosystému.

\section{Angular: Komplexný sanitizačný systém}
Angular pristupuje k problematike XSS komplexnejšie než React, s detailne prepracovaným sanitizačným systémom. Tento framework rozlišuje päť rôznych kontextov, v ktorých môže byť dynamický obsah vložený: HTML, STYLE, SCRIPT, URL a RESOURCE URL. Pre každý z týchto kontextov aplikuje špecifické sanitizačné pravidlá.

Angularov \texttt{DomSanitizer} service poskytuje API, ktoré umožňuje vývojárom explicitne označiť obsah ako bezpečný pre konkrétny kontext:

\begin{lstlisting}[language=TypeScript]
import { DomSanitizer } from '@angular/platform-browser';

constructor(private sanitizer: DomSanitizer) {}

getSafeHtml() {
  return this.sanitizer.bypassSecurityTrustHtml('<b>Tucny text</b>');
}
\end{lstlisting}

Angular tiež implementuje AOT (Ahead-of-Time) kompiláciu, ktorá počas zostavovania aplikácie analyzuje šablóny a detekuje potenciálne bezpečnostné problémy. Tento prístup umožňuje zachytiť mnohé zraniteľnosti ešte pred nasadením aplikácie.

Napriek týmto robustným mechanizmom má Angular niekoľko obmedzení. Jeho sanitizačný systém môže byť pre vývojárov mätúci, čo môže viesť k chybnému použitiu API. Navyše, v niektorých špeciálnych prípadoch môže byť escapovanie neúplné, najmä pri komplexných kombináciách rôznych kontextov.

\section{Vue.js: Vyvážený prístup s explicitnými direktívami}
Vue.js zaujíma strednú cestu medzi Reactom a Angularom v prístupe k XSS ochrane. Štandardná interpolácia pomocou dvojitých zložených zátvoriek \texttt{\{\{\}\}} automaticky escapuje všetok dynamický obsah, podobne ako React.

\begin{lstlisting}[language=HTML]
<template>
  <div>{{ userInput }}</div> <!-- Bezpecne - obsah je escapovany -->
</template>
\end{lstlisting}

Pre prípady, kedy je potrebné vložiť HTML obsah, Vue.js poskytuje direktívu \texttt{v-html}, ktorá slúži podobne ako Reactové \texttt{dangerouslySetInnerHTML}:

\begin{lstlisting}[language=HTML]
<template>
  <div v-html="userInput"></div> <!-- Potencialne nebezpecne -->
</template>
\end{lstlisting}

Vue.js tiež implementuje špeciálne ošetrenie pre atribúty, kde automaticky escapuje potenciálne nebezpečné hodnoty. Napríklad pri bindovaní atribútov pomocou \texttt{v-bind} alebo skrátenej syntaxe \texttt{:}, Vue aplikuje kontextovo závislé escapovanie.

Špeciálnym prípadom sú JavaScriptové URL, kde Vue poskytuje lepšiu ochranu ako React, ale stále nie je dokonalá:

\begin{lstlisting}[language=HTML]
<template>
  <a :href="userUrl">Klikni sem</a> <!-- Ciastocne chranene -->
</template>
\end{lstlisting}

Vue tiež ponúka možnosť definovať vlastné sanitizačné funkcie, čo umožňuje vývojárom prispôsobiť ochranu špecifickým potrebám aplikácie.

\section{Svelte: Kompilačný prístup k bezpečnosti}
Svelte pristupuje k problematike XSS unikátnym spôsobom, ktorý vychádza z jeho kompilačného modelu. Namiesto toho, aby sa sanitizácia vykonávala počas behu aplikácie, Svelte generuje optimalizovaný kód, ktorý už obsahuje vstavané bezpečnostné mechanizmy.

Štandardná interpolácia v Svelte automaticky escapuje obsah:

\begin{lstlisting}[language=HTML]
<script>
  let userInput = '<script>alert("XSS")<\/script>';
</script>

<p>{userInput}</p> <!-- Bezpecne - obsah je escapovany -->
\end{lstlisting}

Pre vkladanie HTML Svelte poskytuje špeciálnu syntax \texttt{@html}, ktorá je explicitne označená ako potenciálne nebezpečná:

\begin{lstlisting}[language=HTML]
<p>{@html userInput}</p> <!-- Potencialne nebezpecne -->
\end{lstlisting}

Svelte tiež implementuje statickú analýzu počas kompilácie, ktorá môže detegovať niektoré zjavné bezpečnostné problémy. Tento prístup umožňuje zachytiť potenciálne zraniteľnosti ešte pred spustením aplikácie.

Avšak, Svelte má niektoré obmedzenia v porovnaní s komplexnejšími frameworkami ako Angular. Chýba mu kontextovo závislé escapovanie a jeho sanitizačný systém nie je tak detailne prepracovaný. Navyše, ako relatívne nový framework, má menšiu komunitu a menej overených bezpečnostných postupov.

\section{Porovnanie s Elmom: Jazykový prístup k bezpečnosti}
Elm sa od vyššie uvedených JavaScriptových frameworkov líši zásadným spôsobom. Kým React, Angular, Vue a Svelte sú knižnice alebo frameworky postavené nad JavaScriptom, Elm je samostatný programovací jazyk s vlastným prostredím a typovým systémom.

Tento rozdiel umožňuje Elm-u implementovať oveľa dôslednejšiu ochranu proti XSS. V Elm-e neexistuje ekvivalent k \texttt{dangerouslySetInnerHTML} alebo \texttt{v-html}. Všetok HTML obsah sa generuje prostredníctvom typovo bezpečných funkcií z modulu \texttt{Html}, ktoré automaticky escapujú všetky dynamické hodnoty.

Napríklad pokus o vloženie potenciálne nebezpečného HTML v Elm-e vyžaduje explicitné použitie špeciálnych funkcií:

\begin{lstlisting}[language=Elm]
view : Model -> Html Msg
view model =
    div [] [ text model.userInput ] -- Vzdy bezpecne
\end{lstlisting}

Elm tiež neumožňuje priamu manipuláciu s DOM, čím eliminuje celú triedu DOM-based XSS útokov. Všetky zmeny v DOM prechádzajú cez Elm runtime, ktorý aplikuje dôsledné bezpečnostné pravidlá.

Ďalšou výhodou Elm-u je jeho typový systém, ktorý explicitne rozlišuje medzi obyčajnými reťazcami a HTML obsahom. Toto rozlíšenie je vynútené na úrovni kompilátora, čím sa výrazne znižuje pravdepodobnosť náhodného vloženia nebezpečného kódu.

\section{Aktuálne mechanizmy ochrany proti XSS v jazyku Elm}

 V porovnaní s tradičnými JavaScript rámcami poskytuje Elm niekoľko vlastností, ktoré ho prirodzene chránia pred XSS útokmi. Medzi najvýznamnejšie patrí:

\begin{itemize}
    \item silný statický typový systém,
    \item imutabilita dát,
    \item čisto funkčný prístup bez vedľajších účinkov,
    \item automatická sanetizácia dynamických údajov pri generovaní HTML.
\end{itemize}

Pri vytváraní HTML v Elm-e sa používa modul \texttt{Html}, ktorý generuje HTML ako čisto typové hodnoty. Všetky dynamické vstupy (napr. text od používateľa) sú automaticky sanitizované, čím sa eliminuje možnosť injektovania skriptu. Ako uvádza oficiálna dokumentácia \parencite{elmguide2023html}, Elm zámerne neponúka jednoduché spôsoby na priame vkladanie surového HTML do DOM-u.

\section{Potenciálne riziká a výzvy}

Napriek inherentným bezpečnostným mechanizmom existujú určité oblasti, kde môže dôjsť k zraniteľnostiam:

\begin{enumerate}
    \item \textbf{Použitie portov na interakciu s JavaScriptom}: Elm umožňuje komunikáciu s JavaScriptom prostredníctvom portov. Ak nie sú tieto porty správne navrhnuté a ošetrené, môžu otvoriť dvere pre XSS útoky. Útočník môže napríklad zneužiť JavaScriptový kód na čítanie alebo zápis do DOM-u, čím obchádza bezpečnostný model Elm-u \parencite{czaplicki2012elm}.

    \item \textbf{Vkladanie nebezpečného HTML obsahu}: Aj keď Elm štandardne neumožňuje priame vkladanie nebezpečného HTML, existujú funkcie ako \texttt{Html.node}, \texttt{Html.lazy} či \texttt{Html.Attributes.property}, ktoré pri nesprávnom použití môžu umožniť injektovanie neescapovaného obsahu \parencite{jfmengels2022vuln}.

    \item \textbf{Ukladanie dát na strane klienta}: Používanie \texttt{localStorage} na ukladanie tokenov môže byť rizikové, pretože tieto údaje sú prístupné z JavaScriptu. V prípade kompromitácie pomocou XSS v externej JavaScript časti môžu byť tieto dáta zneužité \parencite{owaspcheatsheet2023}.
\end{enumerate}

\section{Záver}
Analýza ukazuje, že moderné JavaScriptové frameworky výrazne zlepšili ochranu proti XSS v porovnaní s čistým JavaScriptom, ale stále majú určité obmedzenia. Väčšina z nich poskytuje spôsoby, ako ochranu obísť, čo môže viesť k zraniteľnostiam, ak vývojári tieto mechanizmy použijú nesprávne.

Elm oproti tomu ponúka dôslednejší a systémovejší prístup k bezpečnosti, kde sú ochranné mechanizmy zabudované priamo do jazyka. Tento prístup minimalizuje priestor pre chyby vývojárov a poskytuje vyššiu úroveň bezpečnosti bez potreby explicitných sanitizačných krokov.

Výber frameworku alebo jazyka by preto mal brať do úvahy nielen vývojovú produktivitu a výkon, ale aj jeho bezpečnostný model a schopnosť efektívne čeliť takým hrozbám, ako sú XSS útoky.