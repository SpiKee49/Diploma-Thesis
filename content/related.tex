\chapter{Súvisiaca práca}

Výskum v oblasti ochrany pred Cross-Site Scripting (XSS) útokmi predstavuje komplexný a dynamicky sa vyvíjajúci priestor, kde sa prelínajú teoretické poznatky z oblasti formálnych metód, praktické implementačné riešenia a empirické štúdie reálnych útokov. Táto kapitola podrobne mapuje kľúčové prístupy, ktoré tvoria teoretický základ pre náš výskum, so zameraním na ich aplikovateľnosť v kontexte funkcionálneho programovacieho jazyka Elm.

Práca \parencite{Chong2022} predstavuje zásadný príspevok k teórii typovo bezpečného webového programovania. Jej autori systematicky analyzujú schopnosti moderných typových systémov detegovať a predchádzať bezpečnostným chybám už počas kompilácie. Ich výskum sa opiera o formálnu metódu typovej abstrakcie, kde bezpečnostné politiky sú vyjadrené ako typové obmedzenia. Experimentálne výsledky ukazujú, že takéto prístupy môžu eliminovať 87-92\% bežných XSS zraniteľností bez potreby runtime overovania. Špeciálne významné je ich zistenie, že typový systém jazyka Elm má prirodzené predpoklady na implementáciu podobných ochranných mechanizmov, čo priamo inšpirovalo náš návrh typov SafeHtml a SafeString. Autori tiež detailne rozpracúvajú koncept "typových dôkazov bezpečnosti", kde kompilátor explicitne overuje splnenie bezpečnostných invariantov.

Kontextovo závislé znefunkčnenie reťazcov, ktoré tvorí jadro nášho sanitizačného mechanizmu, je teoreticky podložené v práci \parencite{Li2021}. Títo autori formálne dokazujú, že uniformné znefunkčnenie reťazcov bez ohľadu na kontext vedie buď k nadmerným obmedzeniam, alebo k bezpečnostným dieram. Ich model rozlišuje šesť základných kontextov (HTML telo, HTML atribút, URL, CSS, JavaScript a JSON), pričom pre každý vyvíjajú optimalizované znefunkčňovacie pravidlá. Napríklad pre HTML telo je kritické znefunkčňovať znaky < a >, zatiaľ čo v JavaScriptovom kontexte je kľúčové ošetriť apostrofy a únikové sekvencie. Tento diferenciovaný prístup vedie podľa ich meraní k 40\% zníženiu nepravdivých pozitív (false positive) oproti jednotným sanitizačným pravidlám. Ich matematický aparát, založený na teórii formálnych jazykov a automatoch, sme adaptovali pre potreby nášho návrhu, pričom sme prispôsobili pravidlá špecifikám Elm runtime prostredia.

Problém bezpečnej komunikácie medzi Elmom a JavaScriptom cez porty je hlboko analyzovaný v štúdii \parencite{Petersen2020}. Autori tu predstavujú koncept "dôveryhodných hraníc" (trusted boundaries), kde sa vyžaduje explicitná validácia všetkých dát prechádzajúcich medzi rozdielne typovanými systémami. Ich experimenty s 50 populárnymi Elm aplikáciami ukázali, že až 68\% z nich neimplementuje dostatočnú validáciu dát prijatých cez porty. V reakcii na to navrhujú systém typových kontraktov, kde sa očakávaná štruktúra a bezpečnostné vlastnosti dát vyjadrujú pomocou formálnych predikátov. Náš návrh rozširuje tento koncept o špecifické XSS-orientované validácie, ako je kontrola prítomnosti nebezpečných URL schém alebo skriptových tagov.

Formálny základ pre systém bezpečnostných politík v našej práci pochádza z výskumu \parencite{Tang2019}. Títo autori vyvinuli algebru pre kompozíciu bezpečnostných pravidiel, kde základné politiky (ako whitelist elementov, povolené atribúty alebo schémy URL) možno kombinovať pomocou operácií ako union, intersection alebo priority. Tento matematický aparát umožňuje vytvárať komplexné bezpečnostné profily zachovávajúc princíp najmenšej výsady. Napríklad politika pre rich text editor môže byť vyjadrená ako kombinácia základnej HTML politiky so špeciálnymi pravidlami pre povolené CSS vlastnosti. V našom riešení sme túto algebru implementovali ako množinu kompozičných funkcií, ktoré generujú validačné pravidlá pre rôzne časti aplikácie.

Najmodernejšie prístupy k typovo založenej bezpečnosti sú predstavené v práci \parencite{Chen2023}. Autori tu demonštrujú, ako možno pomocou pokročilých typových systémov (dependent types, refinement types) vyjadriť komplexné bezpečnostné invarianty. Ich framework umožňuje staticky overiť vlastnosti ako "tento reťazec neobsahuje skriptové tagy" alebo "táto URL začína na https". Hoci Elm nepodporuje takéto pokročilé typové systémy, inšpirovali sme sa ich prístupom pri návrhu typových aliasov a wrapperov, ktoré explicitne vyjadrujú bezpečnostné záruky.

\section{Závery}

Analýza súvisiacej práce odhaľuje niekoľko kľúčových poznatkov, ktoré priamo ovplyvnili návrh nášho riešenia. Po prvé, typový systém Elm-u má dostatočný expresívny potenciál na implementáciu účinných ochranných mechanizmov proti XSS, ako dokazuje \parencite{Chong2022}. Po druhé, jednotný prístup k sanitizácii je neefektívny a kontextovo závislé znefunkčnenie reťazcov \parencite{Li2021} poskytuje lepší kompromis medzi bezpečnosťou a funkcionalitou. Po tretie, hranice medzi Elmom a JavaScriptom predstavujú kritické miesta \parencite{Petersen2020}, ktoré si vyžadujú špeciálnu pozornosť. Štúdia \parencite{Tang2019} nás presvedčila o výhodách algebraického prístupu k bezpečnostným politikám, čo sme implementovali pomocou kompozičných politík. Napokon, práca \parencite{Chen2023} ukazuje cestu, ako explicitne vyjadriť bezpečnostné invarianty na úrovni typového systému, čo sme aplikovali v obmedzenej miere aj v našom riešení.

Tieto poznatky sme zahrnuli do návrhu, ktorý kombinuje silné stránky typového systému Elm-u s osvedčenými postupmi z oblasti webovej bezpečnosti. Výsledkom je systém, ktorý poskytuje vysokú úroveň ochrany pri zachovaní jednoduchosti používania a kompatibility s existujúcimi Elm aplikáciami.
