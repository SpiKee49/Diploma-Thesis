\chapter{Výhľady na pokračovanie v práci}

Implementácia middleware pre ochranu proti XSS útokom v jazyku Elm, ktorú sme predstavili v predchádzajúcej kapitole, predstavuje funkčný základ riešenia. V tejto kapitole načrtneme kroky, ktoré sú potrebné na dokončenie práce -- overenie riešenia, identifikáciu jeho obmedzení, návrhy na vylepšenie a zhodnotenie celkového prínosu.

\section{Overenie funkčnosti riešenia}

Kľúčovou otázkou po implementácii akéhokoľvek bezpečnostného riešenia je: \textit{skutočne chráni pred tým, pred čím má chrániť?} Pre overenie funkčnosti našej knižnice je potrebné vykonať systematické testovanie na viacerých úrovniach.

\subsection{Testovanie jednotlivých bezpečnostných politík}

Každá zo štyroch implementovaných bezpečnostných politík musí byť otestovaná s reprezentatívnou množinou XSS útočných vektorov. Plánujeme vytvoriť testovaciu maticu, ktorá systematicky pokryje:

\begin{itemize}
    \item \textbf{AllowTextOnly:} Overenie, že všetky HTML značky a atribúty sú odstránené, pričom čistý textový obsah zostane zachovaný. Testovanie na útočných kódoch ako \texttt{<script>}, \texttt{<img onerror=>}, inline event handlery.

    \item \textbf{AllowSafeHtml:} Validácia, že nebezpečné elementy ako \texttt{<script>}, \texttt{<iframe>}, \texttt{<object>} sú odstránené, ale bezpečné formátovacie značky ako \texttt{<b>}, \texttt{<i>}, \texttt{<p>} zostávajú funkčné. Testovanie mutation XSS a vnorených útočných vektorov.

    \item \textbf{AllowUrl:} Verifikácia, že nebezpečné protokoly ako \texttt{javascript:}, \texttt{data:text/html}, \texttt{vbscript:} sú blokované, zatiaľ čo legitímne \texttt{http:}, \texttt{https:}, \texttt{mailto:} sú povolené.

    \item \textbf{Passthrough:} Konfirmácia, že politika skutočne prenáša dáta bez zmeny pre typy, ktoré to vyžadujú (čísla, boolean hodnoty, JSON objekty).
\end{itemize}

\subsection{Testovanie s reálnymi XSS útočnými databázami}

Pre komplexné overenie plánujeme využiť existujúce, verejne dostupné XSS payloady:

\begin{itemize}
    \item \textbf{XSS Filter Evasion Cheat Sheet} -- obsahuje stovky variácií XSS útokov
    \item \textbf{PayloadsAllTheThings} -- moderná, udržiavaná databáza
    \item \textbf{OWASP XSS Attack Vectors} -- overené útoky zo security auditov
\end{itemize}

Tieto payloady budú systematicky testované proti každej politike. Cieľom nie je 100\% ochrana (čo je v praxi nedosiahnuteľné), ale demonštrácia, že bežné a známe útoky sú efektívne zablokované.

\subsection{Porovnanie s existujúcimi riešeniami}

Pre kontextualizáciu efektívnosti našej knižnice plánujeme vykonať porovnávacie testy s:

\begin{itemize}
    \item \textbf{DOMPurify} (JavaScript) -- referenčná implementácia HTML sanitizácie
    \item \textbf{React's dangerouslySetInnerHTML} -- ako baseline úroveň ochrany v moderných frameworkoch
    \item \textbf{Angular's DomSanitizer} -- príklad kontextovo závislej sanitizácie
\end{itemize}

Výsledky budú prezentované v tabuľke ukazujúcej, koľko percent známych útokov každé riešenie zablokuje pre jednotlivé kategórie (reflected XSS, stored XSS, DOM-based XSS).

\section{Identifikované výhrady a obmedzenia}

Žiadne bezpečnostné riešenie nie je dokonalé. Pre akademickú integritu je kritické explicitne pomenovať obmedzenia našej implementácie.

\subsection{Obmedzenia aktuálnej implementácie sanitizačného enginu}

\textbf{1. Primitívna sanitizácia AllowSafeHtml:}

Aktuálna implementácia politiky \texttt{AllowSafeHtml} používa jednoduchú regex-based sanitizáciu, ktorá odstraňuje iba \texttt{<script>} značky:

\begin{lstlisting}[language=Elm, caption={Súčasná implementácia AllowSafeHtml}, label={lst:current-safehtml}]
AllowSafeHtml ->
    Regex.replace 
        (Regex.fromString "<script.*?</script>" 
            |> Maybe.withDefault Regex.never)
        (\_ -> "")
        rawString
\end{lstlisting}

Táto implementácia \textbf{nezabezpečuje} proti:
\begin{itemize}
    \item Event handler atribútom (\texttt{onload}, \texttt{onerror}, \texttt{onclick}, atď.)
    \item Nebezpečným značkám ako \texttt{<iframe>}, \texttt{<object>}, \texttt{<embed>}
    \item CSS-based XSS cez \texttt{<style>} alebo inline \texttt{style} atribúty
    \item SVG-based XSS cez \texttt{<svg>} elementy
    \item Mutation XSS (mXSS) útokom
\end{itemize}

\textbf{2. Nedostatok HTML parsera:}

Použitie regulárnych výrazov pre sanitizáciu HTML je známe ako problematické \cite{stackoverflow-regex-html}. HTML je kontextovo závislý jazyk, ktorý nemožno spoľahlivo spracovať regulárnymi výrazmi. Potrebujeme skutočný HTML parser, ktorý:
\begin{itemize}
    \item Analyzuje štruktúru HTML stromu
    \item Rozozná validné vs. nevalidné značky
    \item Identifikuje nebezpečné kombinácie atribútov
    \item Správne ošetruje vnorené kontexty
\end{itemize}

\textbf{3. Chybná validácia URL:}

Aktuálna implementácia politiky \texttt{AllowUrl} iba kontroluje, či URL nezačína na \texttt{javascript:}:

\begin{lstlisting}[language=Elm, caption={Súčasná implementácia AllowUrl}, label={lst:current-url}]
AllowUrl ->
    if String.startsWith "javascript:" (String.toLower rawString) then
        ""
    else
        rawString
\end{lstlisting}

Táto implementácia \textbf{ignoruje}:
\begin{itemize}
    \item \texttt{data:} URI scheme, ktoré môže obsahovať HTML
    \item \texttt{vbscript:} pre staršie verzie Internet Explorer
    \item URL encoding obfuscation (\texttt{ja\%76ascript:})
    \item Medzery a null byte injection
    \item Relatívne vs. absolútne URL validáciu
\end{itemize}

\subsection{Architektonické obmedzenia}

\textbf{1. Žiadna ochrana vstupného vektora:}

Ako sme diskutovali v implementačnej kapitole, naše riešenie sa zameriava výhradne na \textit{výstupný vektor} (Elm $\rightarrow$ JS). Dáta prichádzajúce do Elmu z JavaScriptu \textit{nie sú} automaticky sanitizované. Ak vývojár tieto dáta nesprávne zobrazí pomocou nebezpečných funkcií v Elme (napr. \texttt{Html.Attributes.property "innerHTML"}), XSS je stále možný.

\textbf{2. Závislosť na disciplíne vývojára:}

Náš middleware nemôže vynútiť svoje použitie. Ak vývojár:
\begin{itemize}
    \item Vytvorí svoj vlastný port modul
    \item Použije \texttt{Passthrough} politiku pre používateľský vstup
    \item Implementuje vlastnú sanitizáciu
\end{itemize}
...bezpečnostné záruky sú stratené.

\textbf{3. Chýbajúca integrácia s Elm package ekosystémom:}

Aktuálne je riešenie standalone. Pre maximálny prínosu by bolo ideálne:
\begin{itemize}
    \item Publikovať ako Elm package
    \item Integrovať s populárnymi Elm routermi a HTTP knižnicami
    \item Poskytnúť vopred nakonfigurované politiky pre bežné use-case
\end{itemize}

\section{Metodológia vyhodnotenia}

Pre objektívne zhodnotenie prínosu našej práce navrhneme nasledujúce metriky:

\subsection{Kvantitatívne metriky}

\begin{table}[h]
    \centering
    \caption{Návrh hodnotiacich metrík}
    \label{tab:evaluation-metrics}
    \begin{tabular}{|l|l|c|}
        \hline
        \textbf{Metrika}     & \textbf{Meranie}                        & \textbf{Cieľ} \\
        \hline
        XSS Detection Rate   & \% blokovaných útokov z testovacej sady & $>$ 95\%      \\
        False Positive Rate  & \% legitímneho obsahu zablokovaného     & $<$ 5\%       \\
        Performance Overhead & Čas sanitizácie vs. baseline            & $<$ 10ms      \\
        API Complexity       & Počet konceptov na naučenie             & $<$ 5         \\
        Lines of Code        & Veľkosť implementácie                   & $<$ 500 LOC   \\
        \hline
    \end{tabular}
\end{table}

\subsection{Kvalitatívne hodnotenie}

\textbf{1. Použiteľnosť (Usability):}
\begin{itemize}
    \item Jasnosť API -- môže začiatočník použiť knižnicu bez hlbokých security znalostí?
    \item Dokumentácia -- sú príklady použitia dostupné a zrozumiteľné?
    \item Error messages -- sú chybové hlášky informatívne?
\end{itemize}

\textbf{2. Integrovateľnosť:}
\begin{itemize}
    \item Dá sa knižnica použiť v existujúcej aplikácii bez refaktoringu?
    \item Je kompatibilná s populárnymi Elm packages?
\end{itemize}

\textbf{3. Udržateľnosť:}
\begin{itemize}
    \item Je kód dobre štruktúrovaný a dokumentovaný?
    \item Sú prítomné testy pre kritické funkcie?
    \item Je jasné, ako pridať novú bezpečnostnú politiku?
\end{itemize}

\section{Rozdelenie práce do kapitol}

Na základe vyššie uvedených bodov navrhneme štruktúru zostávajúcich kapitol diplomové práce:

\subsection{Kapitola: Testovanie a vyhodnotenie}

Táto kapitola by mala obsahovať:

\begin{enumerate}
    \item \textbf{Testovacia metodológia} -- opis prístupu k testovaniu
    \item \textbf{Testovacia sada} -- zoznam použitých XSS payloadov
    \item \textbf{Výsledky testov} -- tabuľky a grafy ukazujúce efektívnosť
    \item \textbf{Porovnanie s alternatívami} -- benchmarking proti DOMPurify, React, Angular
    \item \textbf{Performance analýza} -- meranie výkonnostného dopadu
\end{enumerate}

\subsection{Kapitola: Diskusia obmedzení a možných vylepšení}

Táto kapitola by mala obsahovať:

\begin{enumerate}
    \item \textbf{Identifikované slabiny} -- detailný rozbor obmedzení (sekcia \ref{sec:limitations})
    \item \textbf{Návrhty riešení} -- konkrétne kroky na odstránenie slabín
    \item \textbf{Trade-offs} -- diskusia o kompromisoch (bezpečnosť vs. výkon, jednoduchosť vs. flexibilita)
    \item \textbf{Alternatívne prístupy} -- čo mohlo byť urobené inak
    \item \textbf{Budúca práca} -- long-term vízia pre vývoj knižnice
\end{enumerate}

\subsection{Kapitola: Záver}

Záverečná kapitola by mala obsahovať:

\begin{enumerate}
    \item \textbf{Zhrnutie prínosu} -- čo práca priniesla
    \item \textbf{Dosiahnuté ciele} -- mapovanie na pôvodne stanovené ciele
    \item \textbf{Praktická aplikovateľnosť} -- kedy a ako použiť riešenie
    \item \textbf{Teoretický príspevok} -- čo práca priniesla do akademickej diskusie
    \item \textbf{Odporúčania} -- pre vývojárov a Elm komunitu
\end{enumerate}

\section{Časový harmonogram dokončenia}

Pre dokončenie práce navrhneme nasledujúci harmonogram:

\begin{table}[h]
    \centering
    \caption{Časový plán dokončenia práce}
    \label{tab:timeline}
    \begin{tabular}{|l|l|l|}
        \hline
        \textbf{Týždeň} & \textbf{Aktivita}             & \textbf{Výstup}    \\
        \hline
        1--2            & Implementácia testovacej sady & 50+ testov         \\
                        & Testovanie politík            & Test report        \\
        \hline
        3               & Porovnanie s alternatívami    & Benchmark report   \\
                        & Performance merania           & Performance report \\
        \hline
        4               & Písanie kapitoly Testovanie   & Kapitola (draft)   \\
        \hline
        5               & Identifikácia obmedzení       & Zoznam slabín      \\
                        & Návrhy vylepšení              & Implementačný plán \\
        \hline
        6               & Písanie kapitoly Diskusia     & Kapitola (draft)   \\
        \hline
        7               & Písanie záverov               & Záverečná kapitola \\
                        & Revízia celej práce           & Kompletný draft    \\
        \hline
        8               & Jazyková korektúra            & Finálna verzia     \\
                        & Príprava prezentácie          & Obhajoba slides    \\
        \hline
    \end{tabular}
\end{table}

\section{Záver kapitoly}

V tejto kapitole sme načrtli komplexný plán na dokončenie diplomovej práce. Identifikovali sme tri kľúčové oblasti, ktoré vyžadujú pozornosť:

\begin{enumerate}
    \item \textbf{Overenie riešenia} -- systematické testovanie funkčnosti a efektívnosti
    \item \textbf{Identifikácia obmedzení} -- čestné pomenovanie slabín a nedostatkov aktuálnej implementácie
    \item \textbf{Návrhy vylepšení} -- konkrétne kroky, ako riešenie posunúť na produkčnú úroveň
\end{enumerate}

V nasledujúcich kapitolách tieto body rozpracujeme do detailnej podoby s konkrétnymi výsledkami, meraniami a závermi.